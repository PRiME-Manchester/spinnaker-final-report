\section{EMC testing documentation}\label{emc-testing-documentation}

\textbf{Links} http://learnemc.com/introduction-to-emc

\subsection{EMC background}\label{emc-background}

\textbf{Electromagnetic compatibility} (\textbf{EMC}) is the branch of
electrical engineering concerned with the unintentional generation,
propagation and reception of electromagnetic energy which may cause
unwanted effects such as
\href{https://en.wikipedia.org/wiki/Electromagnetic_interference}{electromagnetic
interference} (EMI) or even physical damage in operational equipment.
The goal of EMC is the correct operation of different equipment in a
common electromagnetic environment.

EMC pursues two main classes of issue. \textbf{Emission} is the
generation of electromagnetic energy, whether deliberate or accidental,
by some source and its release into the environment. EMC studies the
unwanted emissions and the countermeasures which may be taken in order
to reduce unwanted emissions. The second class, \textbf{susceptibility}
is the tendency of electrical equipment, referred to as the victim, to
malfunction or break down in the presence of unwanted emissions, which
are known as Radio frequency interference (RFI). \textbf{Immunity} is
the opposite of susceptibility, being the ability of equipment to
function correctly in the presence of RFI, with the discipline of
``hardening'' equipment being known equally as susceptibility or
immunity. A third class studied is \textbf{coupling}, which is the
mechanism by which emitted interference reaches the victim.

Interference mitigation and hence electromagnetic compatibility may be
achieved by addressing any or all of these issues, i.e., quieting the
sources of interference, inhibiting coupling paths and/or hardening the
potential victims. In practice, many of the engineering techniques used,
such as grounding and shielding, apply to all three issues.

\subsection{Software description}\label{software-description}

\section{Power measurement on the SpiNN4
board}\label{power-measurement-on-the-spinn4-board}

\subsection{Hardware used}\label{hardware-used}

\subsubsection{ADC Arduino shield}\label{adc-arduino-shield}

\subsubsection{Shunt amplifier circuits}\label{shunt-amplifier-circuits}

\includegraphics{https://lh3.googleusercontent.com/7izGFqDQa3kOqovWX3c2nPbZOhT5zpO-8J2_b6FWCSzR_Gd55gu7GIugUi8jaAwS1BrpuQ=s200}
Shunt amplifier

\includegraphics{https://lh3.googleusercontent.com/lLw3qGgwjhGo6hZ8DHrkl-ya9C_N1Z3O2uiNqdedV3h4bQVoiRa_7sy2nLpjBbwm5UWO6Q=s200}
Shunt amplifier

\includegraphics{https://lh3.googleusercontent.com/ciFtScA-Sh6h7DyY__3vllkZBxYeuPN4YwdKTVOXZhl3T-ujDx0UGp85T_k6G-5SvhVxfA=s300}
Shunt amplifiers mounted on SpiNN4 board

\includegraphics{https://lh3.googleusercontent.com/msSIQWgu1V_ONSZs12Yf98xhnSLv61XmQmKj-TKl25sA0g5owrfCj21V931x3EURm83UoQ=s500}
SpiNN4 board showing the shunt amplifiers and the ADC Arduino shield

\includegraphics{https://lh3.googleusercontent.com/8JgsV0bi3ZmOY9fCbFOhBAV0lXHeYw1FmeKOhjtksmhWU8y4HXUNSCPvphFVv5r7NHs0vA=s500}
Close-up of the ADC Arduino shield

\subsection{Software used}\label{software-used}

\subsubsection{Arduino Processing
software}\label{arduino-processing-software}

\href{http://processing.org/}{Processing} is an open source language/
development tool for writing programs in \emph{other}computers. Useful
when you want those other computers to ``talk'' with an Arduino, for
instance to display or save some data collected by the Arduino.

\textbf{Processing} is an
\href{https://en.wikipedia.org/wiki/Open-source_software}{open source}
\href{https://en.wikipedia.org/wiki/Programming_language}{programming
language} and
\href{https://en.wikipedia.org/wiki/Integrated_development_environment}{integrated
development environment} (IDE) built for the electronic arts,
\href{https://en.wikipedia.org/wiki/New_media_art}{new media art}, and
\href{https://en.wikipedia.org/wiki/Visual_design}{visual design}
communities with the purpose of teaching the fundamentals of
\href{https://en.wikipedia.org/wiki/Computer_programming}{computer
programming} in a visual context, and to serve as the foundation for
electronic sketchbooks. The project was initiated in 2001 by
\href{https://en.wikipedia.org/wiki/C.E.B._Reas}{Casey Reas} and
\href{https://en.wikipedia.org/wiki/Benjamin_Fry}{Benjamin Fry}, both
formerly of the Aesthetics and Computation Group at the
\href{https://en.wikipedia.org/wiki/MIT_Media_Lab}{MIT Media Lab}. One
of the stated aims of Processing is to act as a tool to get
non-programmers started with programming, through the instant
gratification of visual feedback. The language builds on the
\href{https://en.wikipedia.org/wiki/Java_(programming_language)}{Java
language}, but uses a simplified syntax and graphics programming model.
In 2012, they started the Processing Foundation along with
\href{https://en.wikipedia.org/wiki/Daniel_Shiffman}{Daniel Shiffman},
who formally joined as a third project lead.

\textbf{Features}

Processing includes a \emph{sketchbook}, a minimal alternative to an
\href{https://en.wikipedia.org/wiki/Integrated_development_environment}{integrated
development environment} (IDE) for organizing projects.

Every Processing sketch is actually a subclass of the
\href{http://processing.googlecode.com/svn/trunk/processing/build/javadoc/core/processing/core/PApplet.html}{PApplet}
\href{https://en.wikipedia.org/wiki/Java_class}{Java class} which
implements most of the Processing language's features.

When programming in Processing, all additional classes defined will be
treated as \href{https://en.wikipedia.org/wiki/Inner_class}{inner
classes} when the code is translated into pure Java before compiling.
This means that the use of
\href{https://en.wikipedia.org/wiki/Static_variables}{static variables}
and \href{https://en.wikipedia.org/wiki/Static_methods}{methods} in
classes is prohibited unless you explicitly tell Processing that you
want to code in pure Java mode.

Processing also allows for users to create their own classes within the
PApplet sketch. This allows for complex
\href{https://en.wikipedia.org/wiki/Data_types}{data types} that can
include any number of arguments and avoids the limitations of solely
using standard data types such as:
\href{https://en.wikipedia.org/wiki/Integer_(computer_science)}{int}
(integer),
\href{https://en.wikipedia.org/wiki/Character_(computing)}{char}
(character), \href{https://en.wikipedia.org/wiki/Floating_point}{float}
(real number), and color (RGB,
\href{https://en.wikipedia.org/wiki/RGBA_color_space}{ARGB}, hex).

\section{Markov Chain Monte Carlo on
SpiNNaker}\label{markov-chain-monte-carlo-on-spinnaker}

https://darrenjw.wordpress.com/2010/04/28/mcmc-programming-in-r-python-java-and-c/

\href{http://en.wikipedia.org/wiki/Markov_chain_Monte_Carlo}{Markov
chain Monte Carlo} (MCMC) is a powerful simulation technique for
exploring high-dimensional probability distributions. It is particularly
useful for exploring posterior probability distributions that arise in
Bayesian statistics. Although there are some generic tools (such as
\href{http://www.mrc-bsu.cam.ac.uk/bugs/}{WinBugs} and
\href{http://www-fis.iarc.fr/~martyn/software/jags/}{JAGS}) for doing
MCMC, for non-standard problems it is often desirable to code up MCMC
algorithms from scratch. It is then natural to wonder what programming
languages might be good for this purpose. There are hundreds of
programming languages one could in principle use for MCMC programming,
but it is necessary to use a language with a good scientific library,
including good random number generation routines. I have used a large
number of programming languages over the years, but these days I mostly
program in \href{http://www.r-project.org/}{R},
\href{http://www.python.org/}{Python},\href{http://java.sun.com/}{Java}
or
\href{http://en.wikipedia.org/wiki/C_\%28programming_language\%29}{C}. I
find each of these languages interesting and useful, with different
strengths and weaknesses, and I find that no one of these languages
dominates any of the others in all situations.

\subsection{Example}\label{example}

For the purposes of this post we will focus on Gibbs sampling for a
simple bivariate distribution defined on the half-plane
\emph{x}\textgreater{}0.

\[
f(x,y) = k x^2 \exp{(-xy^2 - y^2 + 2^y - 4x)}
\]

The statistical motivation is not important for this post, but this is
the kind of distribution which arises naturally in Bayesian inference
for the mean and variance of a normal random sample. Gibbs sampling is a
simple MCMC scheme which samples in turn from the full-conditional
distributions. In this case, the full-conditional for \emph{x} is
\(\Gamma(3, y^2+4)\) and the full-conditional for \emph{y} is
N(1/(\emph{x}+1), 1/(\emph{x}+1)). The resulting Gibbs sampling
algorithm is very simple and works very efficiently, but for
illustrative purposes, we will consider generating 20,000 iterations
with a ``thin'' of 500 iterations.

\subsubsection{C}\label{c}

The language I have used most for the development of MCMC algorithms is
C (usually strict ANSI C). C is fast and efficient, and gives the
programmer lots of control over how computations are carried out.
The\href{http://www.gnu.org/software/gsl/}{GSL} is an excellent
scientific library for C, which includes good random number generation
facilities. A program for this problem is given below.

\begin{verbatim}
#include <math.h>
#include <stdlib.h>
#include <gsl/gsl_rng.h>
#include <gsl/gsl_randist.h>
 
void main()
{
  int N=20000;
  int thin=500;
  int i,j;
  gsl_rng *r = gsl_rng_alloc(gsl_rng_mt19937);
  double x=0;
  double y=0;
  printf("Iter x y\n");
  for (i=0;i<N;i++) {
    for (j=0;j<thin;j++) {
      x=gsl_ran_gamma(r,3.0,1.0/(y*y+4));
      y=1.0/(x+1)+gsl_ran_gaussian(r,1.0/sqrt(x+1));
    }
    printf("%d %f %f\n",i,x,y);
  }
}
\end{verbatim}

This can be compiled and run (on Linux) with commands like:

\begin{verbatim}
gcc -O2 -lgsl -lgslcblas gibbs.c -o gibbs
time ./gibbs > data.tab
\end{verbatim}

\textbf{PCG}

http://www.pcg-random.org/

PCG is a family of simple fast space-efficient statistically good
algorithms for random number generation. Unlike many general-purpose
RNGs, they are also hard to predict.

\includegraphics{https://lh3.googleusercontent.com/_jUd0J5vJlEIKPKaLEhj__JXJpugfrL8pkKAutjpmWQi0yu6NQOUy4LiE7JcTU0tlJcaqQ=s700}
PCG family comparison

\begin{itemize}
\itemsep1pt\parskip0pt\parsep0pt
\item
  For the PCG family, arbitrary \emph{k}-dimensional equidistribution
  (and the huge periods it implies) requires PCG's extended generation
  scheme.
\end{itemize}

† ChaCha entry based on
\href{https://gist.github.com/orlp/32f5d1b631ab092608b1}{an optimized
C++ implementation} of ChaCha, kindly provided by Orson Peters.

In \href{https://en.wikipedia.org/wiki/Statistics}{statistics},
\textbf{Markov chain Monte Carlo} (\textbf{MCMC}) methods are a class of
\href{https://en.wikipedia.org/wiki/Algorithm}{algorithms} for sampling
from a
\href{https://en.wikipedia.org/wiki/Probability_distribution}{probability
distribution} based on constructing a
\href{https://en.wikipedia.org/wiki/Markov_chain}{Markov chain} that has
the desired distribution as its
\href{https://en.wikipedia.org/wiki/Markov_chain\#Steady-state_analysis_and_limiting_distributions}{equilibrium
distribution}. The state of the chain after a number of steps is then
used as a sample of the desired distribution. The quality of the sample
improves as a function of the number of steps.

\begin{figure}[htbp]
\centering
\includegraphics{https://lh3.googleusercontent.com/qdgFOWPvEoEfV9_a0yRfnGP6uIdNtq7obhHyWWpp48WSb2ej-r4DhsqeyjrpwxVnIW5L8Q=s500}
\caption{enter image description here}
\end{figure}

Convergence of the
\href{https://en.wikipedia.org/wiki/Metropolis-Hastings_algorithm}{Metropolis-Hastings
algorithm}. MCMC attempts to approximate the blue distribution with the
orange distribution

\textbf{Random walk Monte Carlo} methods make up a large subclass of
MCMC methods.

\subsection{Application domains}\label{application-domains}

\begin{itemize}
\item
  MCMC methods are primarily used for calculating
  \href{https://en.wikipedia.org/wiki/Numerical_analysis}{numerical
  approximations} of
  \href{https://en.wikipedia.org/wiki/Multiple_integral}{multi-dimensional
  integrals}, for example in
  \href{https://en.wikipedia.org/wiki/Bayesian_statistics}{Bayesian
  statistics},
  \href{https://en.wikipedia.org/wiki/Computational_physics}{computational
  physics},
  \href{https://en.wikipedia.org/wiki/Computational_biology}{computational
  biology} and
  \href{https://en.wikipedia.org/wiki/Computational_linguistics}{computational
  linguistics}.\href{https://en.wikipedia.org/wiki/Markov_chain_Monte_Carlo\#cite_note-1}{{[}1{]}}\href{https://en.wikipedia.org/wiki/Markov_chain_Monte_Carlo\#cite_note-2}{{[}2{]}}
\item
  In \href{https://en.wikipedia.org/wiki/Bayesian_statistics}{Bayesian
  statistics}, the recent development of MCMC methods has been a key
  step in making it possible to compute large
  \href{https://en.wikipedia.org/wiki/Bayesian_network\#Hierarchical_models}{hierarchical
  models} that require integrations over hundreds or even thousands of
  unknown
  parameters.\href{https://en.wikipedia.org/wiki/Markov_chain_Monte_Carlo\#cite_note-3}{{[}3{]}}
\item
  They are also used for generating samples that gradually populate the
  rare failure region in
  \href{https://en.wikipedia.org/wiki/Rare_event_sampling}{rare event
  sampling}.
\end{itemize}

\subsection{Classification}\label{classification}

\subsubsection{Random walk Monte Carlo
methods}\label{random-walk-monte-carlo-methods}

\paragraph{Multi-dimensional
integrals}\label{multi-dimensional-integrals}

When an MCMC method is used for approximating a
\href{https://en.wikipedia.org/wiki/Multiple_integral}{multi-dimensional
integral}, an
\href{https://en.wikipedia.org/wiki/Statistical_ensemble}{ensemble} of
``walkers'' move around randomly. At each point where a walker steps,
the integrand value at that point is counted towards the integral. The
walker then may make a number of tentative steps around the area,
looking for a place with a reasonably high contribution to the integral
to move into next.

Random walk Monte Carlo methods are a kind of random
\href{https://en.wikipedia.org/wiki/Computer_simulation}{simulation} or
\href{https://en.wikipedia.org/wiki/Monte_Carlo_method}{Monte Carlo
method}. However, whereas the random samples of the integrand used in a
conventional
\href{https://en.wikipedia.org/wiki/Monte_Carlo_integration}{Monte Carlo
integration} are
\href{https://en.wikipedia.org/wiki/Statistically_independent}{statistically
independent}, those used in MCMC methods are
*\href{https://en.wikipedia.org/wiki/Correlation}{correlate*d}. A
\href{https://en.wikipedia.org/wiki/Markov_chain}{Markov chain} is
constructed in such a way as to have the integrand as its
\href{https://en.wikipedia.org/wiki/Markov_chain\#Steady-state_analysis_and_limiting_distributions}{equilibrium
distribution}.

\paragraph{Examples}\label{examples}

Examples of random walk Monte Carlo methods include the following:

\begin{itemize}
\item
  \href{https://en.wikipedia.org/wiki/Metropolis\%E2\%80\%93Hastings_algorithm}{Metropolis--Hastings
  algorithm}: This method generates a
  \href{https://en.wikipedia.org/wiki/Random_walk}{random walk} using a
  proposal density and a method for rejecting some of the proposed
  moves.
\item
  \href{https://en.wikipedia.org/wiki/Gibbs_sampling}{Gibbs sampling}:
  This method requires all the
  \href{https://en.wikipedia.org/wiki/Conditional_distribution}{conditional
  distributions} of the target distribution to be sampled exactly. When
  drawing from the full-conditional distributions is not straightforward
  other samplers-within-Gibbs are used (e.g, see
  \href{https://en.wikipedia.org/wiki/Markov_chain_Monte_Carlo\#cite_note-4}{{[}4{]}}\href{https://en.wikipedia.org/wiki/Markov_chain_Monte_Carlo\#cite_note-5}{{[}5{]}}\href{https://en.wikipedia.org/wiki/Markov_chain_Monte_Carlo\#cite_note-:0-6}{{[}6{]}}).
  Gibbs sampling is popular partly because it does not require any
  `tuning'.
\item
  \href{https://en.wikipedia.org/wiki/Slice_sampling}{Slice sampling}:
  This method depends on the principle that one can sample from a
  distribution by sampling uniformly from the region under the plot of
  its density function. It alternates uniform sampling in the vertical
  direction with uniform sampling from the horizontal `slice' defined by
  the current vertical position.
\item
  \href{https://en.wikipedia.org/wiki/Multiple-try_Metropolis}{Multiple-try
  Metropolis}: This method is a variation of the Metropolis--Hastings
  algorithm that allows multiple trials at each point. By making it
  possible to take larger steps at each iteration, it helps address the
  \href{https://en.wikipedia.org/wiki/Curse_of_dimensionality}{curse of
  dimensionality}.\href{https://en.wikipedia.org/wiki/Markov_chain_Monte_Carlo\#cite_note-7}{{[}7{]}}\href{https://en.wikipedia.org/wiki/Markov_chain_Monte_Carlo\#cite_note-8}{{[}8{]}}
\item
  \href{https://en.wikipedia.org/wiki/Reversible-jump}{Reversible-jump}:
  This method is a variant of the Metropolis--Hastings algorithm that
  allows proposals that change the dimensionality of the
  space.\href{https://en.wikipedia.org/wiki/Markov_chain_Monte_Carlo\#cite_note-9}{{[}9{]}}
  MCMC methods that change dimensionality have long been used in
  \href{https://en.wikipedia.org/wiki/Statistical_physics}{statistical
  physics} applications, where for some problems a distribution that is
  a \href{https://en.wikipedia.org/wiki/Grand_canonical_ensemble}{grand
  canonical ensemble} is used (e.g., when the number of molecules in a
  box is variable). But the reversible-jump variant is useful when doing
  MCMC or Gibbs sampling over
  \href{https://en.wikipedia.org/wiki/Nonparametric}{nonparametric}
  Bayesian models such as those involving the
  \href{https://en.wikipedia.org/wiki/Dirichlet_process}{Dirichlet
  process} or
  \href{https://en.wikipedia.org/wiki/Chinese_restaurant_process}{Chinese
  restaurant process}, where the number of mixing
  components/clusters/etc. is automatically inferred from the data.
\end{itemize}

\section{Calibrating the SpiNNaker temperature
sensors}\label{calibrating-the-spinnaker-temperature-sensors}

\begin{figure}[htbp]
\centering
\includegraphics{https://lh3.googleusercontent.com/o9NRlj75RoJSTpS2LiKeDNq1Oijs0ZzPmF6bq0QHhey83qXs2yvxCQa4F3e8XO3vJtYy_A=s700}
\caption{enter image description here}
\end{figure}

\section{Event-based camera on cheap smartphone sensors and programmable
hardware}\label{event-based-camera-on-cheap-smartphone-sensors-and-programmable-hardware}

\subsection{Camera}\label{camera}

\includegraphics{https://lh3.googleusercontent.com/gG2gAYbSC5kXQN-gpJ4AwolR-f7DfwRPTEwB2Q4abScEDAEqdSaGL0ESBKkoP6nfz9aXnQ=s300}
OV7670

\includegraphics{https://lh3.googleusercontent.com/SB8xW5e38Cx87is1yodl2YpyQ8Ef-a2peA4RxucaHjZhnfxs6FJ8MZDCJeG4Lh1y_Lmuyw=s300}
OV5642

\subsection{OV7670}\label{ov7670}

\subsubsection{Introduction}\label{introduction}

OV7670 image sensor, small volume, low operating voltage, providing all
functions of a single chip of VGA camera and image processor. Through
SCCB bus control, the sensor can output the whole frame, sampling, and
various resolution 8 bits of data. The product VGA image can reach up to
a maximum of 30 frames per second. Users can completely control the
image quality, data format and transmission mode. All the process of
image processing functions can be through the SCCB programming
interface, including gamma curve, white balance, saturation and chroma .

OmmiVision image sensor has been in application of unique sensor
technology, by reducing or eliminating the optical or electronic defect
such as fixed pattern noise, tail, floating away, etc., to improve the
quality of the image, and get the clear and stable color images.

\textbf{Specifications:}

\begin{itemize}
\itemsep1pt\parskip0pt\parsep0pt
\item
  Photosensitive array: 640x480
\item
  IO Voltage: 2.5 V to 3.0 V (internal LDO for nuclear power 1.8V)
\item
  Power operation: 60 mW/15 fps
\item
  Sleep: \textless{}20 μA
\item
  Temperature Operating: -30 ℃ to 70 ℃
\item
  Stable: 0 ℃ to 50 ℃
\item
  Output Formats (8): YUV/YCbCr4: 2:2 RGB565/555/444 GRB4: 2:2 Raw RGB
  Data
\item
  Optical size: 1/6 "
\item
  FOV: 25 °
\item
  Maximum frame rate: 30 fps VGA
\item
  Sensitivity: 1.3 V/(Lux-sec)
\item
  SNR: 46 dB
\item
  Dynamic range: 52 dB
\item
  View Mode: Progressive
\item
  Electronic Exposure: 1 line to 510 line
\item
  Pixel Size: 3.6 μm x 3.6 μm
\item
  Dark current: 12 mV / s at 60 ℃
\end{itemize}

\textbf{Features:}

\begin{itemize}
\itemsep1pt\parskip0pt\parsep0pt
\item
  High sensitivity suitable for illumination applications
\item
  Low voltage suitable for embedded applications
\item
  Standard SCCB interface compatible with I2C interface
\item
  RawRGB, RGB (GRB4:2:2, RGB565/555/444), YUV (4:2:2) and YCbCr (4:2:2)
  output format
\item
  Supports VGA, CIF, and from a variety of sizes CIF to 40x30
\item
  VarioPixel sub-sampling mode
\item
  ISP has a compensation function to eliminate noise and dead pixels
\item
  Support for image scaling
\item
  Compensation for loss of optical lens
\item
  50/60Hz automatic detection
\item
  Saturation automatically adjust (UV adjustment)
\item
  Automatically adjust edge enhancement
\item
  Automatically adjust the noise reduction
\item
  Automatically affect the control functions include: automatic exposure
  control, automatic gain control, automatic white balance, automatic
  elimination of light stripes, automatic black level calibration image
  quality control including color saturation, hue, gamma, sharpness
\end{itemize}

Read more:
http://www.elecfreaks.com/store/ov7670-camera-module-p-705.html\#ixzz3xzHrceGf

\subsection{OV5642}\label{ov5642}

\subsubsection{Introduction}\label{introduction-1}

The OV5642 (color) image sensor is a low voltage, high-performance,
1/4-inch 5 megapixel CMOS image sensor that provides the full
functionality of a single chip 5 megapixel (2592x1944) camera using
OmniBSI™ technology in a small footprint package. It provides
full-frame, sub-sampled, windowed or arbitrarily scaled 8-bit/10-bit
images in various formats via the control of the Serial Camera Control
Bus (SCCB) interface or MIPI interface. The OV5642 has an image array
capable of operating at up to 15 frames per second (fps) in 5 megapixel
resolution with complete user control over image quality, formatting and
output data transfer. All required image processing functions, including
exposure control, gamma, white balance, color saturation, hue control,
defective pixel canceling, noise canceling, etc., are programmable
through the SCCB interface, MIPI interface or embedded microcontroller.
The OV5642 also includes a compression engine for increased processing
power. In addition, Omnivision image sensors use proprietary sensor
technology to improve image quality by reducing or eliminating common
lighting/electrical sources of image contamination, such as fixed
pattern noise, smearing, etc., to produce a clean, fully stable, color
image. The OV5642 has an embedded microcontroller, which can be combined
with an internal autofocus engine and programmable general purpose I/O
modules (GPIO) for external autofocus control. It also provides an
anti-shake function with an internal anti-shake engine. For
identification and storage purposes, the OV5642 also includes a one-time
programmable (OTP) memory. Compared to its predecessor, the OV5642 has
embedded TrueFocus™ Lite that enables extended depth of field (EDoF).
The OV5642 supports both a digital video parallel port and a serial MIPI
port. The MIPI and ISP interface can be used for a second camera sensor
without requiring a dual serial port camera system.

\textbf{Features:}

\begin{itemize}
\itemsep1pt\parskip0pt\parsep0pt
\item
  1.4 μm OmniBSI technology
\item
  ultra high performance
\item
  embedded TrueFocus ISP enabling better denoise, sharpening, gamma
  correction and colour correction
\item
  automatic image control functions:

  \begin{itemize}
  \itemsep1pt\parskip0pt\parsep0pt
  \item
    automatic exposure control (AEC)
  \item
    automatic white balance (AWB)
  \item
    automatic band filter (ABF)
  \item
    automatic 50/60 Hz luminance detection
  \item
    automatic black level calibration (ABLC)
  \end{itemize}
\item
  programmable controls for frame rate, AEC/AGC 16-zone
  size/position/weight control, mirror and flip, scaling, cropping,
  windowing, and panning
\item
  image quality controls: color saturation, hue, gamma, sharpness (edge
  enhancement), lens correction, defective pixel canceling, and noise
  canceling
\item
  support for output formats: RAW RGB, RGB565/555/444, CCIR656,
  YUV422/420, YCbCr422, and compression
\item
  support for auto focus control (AFC)
\item
  standard serial SCCB interface
\item
  MIPI serieal input and output interface
\item
  programmable I/O drive capability
\item
  support for mechanical shutter, ND filter and IRIS control
\item
  built-in 1.5V regulator for core
\end{itemize}

\textbf{Specifications:}

\begin{itemize}
\itemsep1pt\parskip0pt\parsep0pt
\item
  active array size: 2592x1944
\item
  power supply:

  \begin{itemize}
  \itemsep1pt\parskip0pt\parsep0pt
  \item
    core: 1.5 V +/-5\% (internal regulator)
  \item
    analog: 2.6 - 3.0 V
  \item
    I/O: 1.71 - 3.0 V
  \end{itemize}
\item
  power requirements:

  \begin{itemize}
  \itemsep1pt\parskip0pt\parsep0pt
  \item
    active: 270 mA
  \item
    standby: 25 μm
  \end{itemize}
\item
  temperature range:

  \begin{itemize}
  \itemsep1pt\parskip0pt\parsep0pt
  \item
    operating: -30 ℃ to 70 ℃
  \item
    stable image: 0 ℃ to 50 ℃
  \end{itemize}
\item
  lens sie: 1/4 "
\item
  lens chief ray angle: 24 deg non-linear
\item
  input clock frequency: 6-54 MHz
\item
  shutter: rolling shutter
\item
  maximum image transfer rate:

  \begin{itemize}
  \itemsep1pt\parskip0pt\parsep0pt
  \item
    5 megapixel (2592x1944): 15 fps (and any size scaling down from 5
    megapixel)
  \item
    1080p (1920x1080): 30 fps
  \item
    720p (1280x720): 60 fps
  \item
    VGA (640x480): 60 fps
  \item
    QVGA (320x240): 120 fps
  \end{itemize}
\item
  sensitivity: 680 mV /(lux-sec)
\item
  S/N ratio: 36 dB
\item
  dynamic range: 68 dB
\item
  pixel size: 1.4 μm x 1.4 μm
\item
  image area: 3673.6 μm x 2738.4 μm
\item
  package dimensions: 6945 μm x 6696 μm
\item
  die dimensions: 6960 μm x 6710 μm
\end{itemize}

\subsection{Zybo/Zedboard
implementation}\label{zybozedboard-implementation}

\begin{center}\rule{0.5\linewidth}{\linethickness}\end{center}

\includegraphics{https://lh3.googleusercontent.com/4_R1nekaDU5RTx3HTAkN_BXW72U1XVhs62EIZYSHSbBcqDwzjlzm28QwWL-kuw3iKN5wKw=s500}
Zybo board

\includegraphics{https://lh3.googleusercontent.com/EXCOSrRNLwdI28QtsVJ57E05t7irgevXtloe8kQQAJjuORpLZKH1_-3ubWxjDZMmYAAzxQ=s500}
Zedboard

\subsection{Vivado design}\label{vivado-design}

\includegraphics{https://lh3.googleusercontent.com/8Rhvx2sQosSkM0PtcvkuCvC-hLaR8IPjrnmu5DeFwXm1eg8hRlfmT1ZR8mrDY1Yd1wBZpA=s800}
Vivado design
